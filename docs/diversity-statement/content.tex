I have long been mindful of the barriers faced by the underrepresented
groups in academia and tech industry. My own experiences of growing up
in an economically- and socially-backward community in rural India,
and later studying at an acclaimed private university with an
overwhelmingly upper-class student population has made me understand
the pernicious effects of exclusion, real or perceived, on one's
academic performance and confidence. Considering that the duty of an
academic is not merely to perform research and provide technical
instruction, but also to engage with the community and aid in the
collective advancement of the society, I believe it is imperative to
address the social impediments to this cause by relentlessly promoting
equity, diversity and inclusion. To this end, I have taken several
initiatives, which I discuss below briefly alongside my plans to
promote diversity and inclusion in academia in future.

I see inclusion as a foundational aspect of an effective teaching
philosophy rather than merely an embellishment. Successful outreach
efforts can help promote diversity in CS classrooms by driving the
enrollment from underrepresented groups, but to sustain the diversity
in long term outreach has to be substantiated by equally vigorous
efforts to promote inclusion in classrooms. It is here that I believe
an instructor can have the most impact. In my capacity as a course
instructor at Purdue this semester, I have taken steps to ensure that
every student feels welcome in the classroom, and included in the
classroom discussion. Among the things I undertook is an effort to
address the gender stereotype threat facing women in CS by carefully
choosing examples, case studies and role models that \emph{prime} the
students with positive stereotypes~\cite{MA07,SG11}.  Another
conscious effort was aimed at disadvantaged students whose education
has suffered due to the lack of resources and unfavorable social
conditions in inner cities and developing world. It is unfair to
expect such students to have the same level of confidence as the
privileged students, and therefore unreasonable to assume that they
will participate in classroom discussion, or take advantage of
learning resources without any external intervention. As an
instructor, I have always strived to include disadvantaged students in
the classroom discussion by fostering a conducive environment where
everyone is encouraged to express their opinion without the fear of
judgment, and every opinion is given a thoughtful consideration. I
have also carefully refrained from speech or action that could
directly or indirectly reinforce informal social hierarchies among
students that contribute to defensive social climate~\cite{BGJ02}. In
the future, I plan to continue these good practices, while also taking
new initiatives to promote equity, inclusion and diversity. In
particular, I would like to constitute a private forum headed by a
non-male TA to address gender-specific issues that might arise when
students work in teams on semester-long course projects. Such
self-organized teams are microcosms of the software industry, hence
often suffer from the same biases as the latter. Institutional efforts
to address such biases are therefore essential if CS classrooms were
to become more inclusive.

I have also long been committed to equity, diversity, and inclusion in
my research, mentorship and professional activities. Over the course
of years, I have developed a positive approach towards research
collaboration and mentorship that emphasizes strengths of my
collaborators while acknowledging and embracing their weaknesses. This
approach has helped me build great professional relationships with my
fellow researchers, particularly my female colleagues. Among my
mentees is a freshman female graduate student, who is now a productive
researcher, and my immediate co-author on a recently published journal
paper.  Beyond the formal channels, I frequently engage my colleagues
in conversations about diversity and inclusion, which I found to be
effective in questioning stereotypes, raising awareness about implicit
biases, and understanding cultural differences. As an assistant
professor in the future, I plan to undertake more initiatives aimed at
fostering a sense of belongingness in grad school among students from
vulnerable groups. In particular, I am committed to the creation of a
forum for Women in Software Engineering (WiSE), which understands and
addresses the issues faced by women in software engineering research,
and works towards making the research community more inclusive. I
would also like to constitute a forum to discuss mental health issues
in academia, which seem to disproportionately affect students from
underrepresented groups. Furthermore, I will continue engaging the
academic community in conversations about empathy, equality, diversity
and inclusion, which I believe are necessary to substantiate the
institutional efforts towards furthering these causes.

